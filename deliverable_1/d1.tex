% !TeX root = d1.tex

\documentclass{article}

\usepackage{xcolor}
\usepackage[margin=0.5in]{geometry} 

\title{MindMerge - Deliverable e 1}
\author{Gabriele Benetti, Gioele Berdardini, Luca Fossa Crescini, Luca Sartore}

\begin{document}
\maketitle


\tableofcontents

\section{SWAT Analysis}
\subsection{Strengths}
\begin{itemize}
    \item \textbf{Operating in an market that is hyped and growing: }
    The Artificial intelligence and LLM (large language models) market is currently on everyone's mouth.
    and is growing fast\dots this would make it easy for our team to find sell the product, and potentially find investors 

    \item \textbf{The workplace is getting more and more digital: }
    In the last decades more and more jobs are getting used to using 

    \item \textbf{: }
    \item \textbf{: }


\end{itemize}
\subsection{Weaknesses}

\subsection{Opportunities}

\subsection{Threads}

\section{Components Diagram}

This section illustrate the component diagram of the system. To improve readability we split the components into several sub
components.
Each sub component has a color assigned to it, and the interfaces that are provided by that specific component are painted with the same color.
This helps to quickly identify which component is responsible for a specific interface.
\newline \newline
The list of sub components is the following:
\begin{itemize}
    \item \textcolor[HTML]{8CC86E}{\textbf{Data Base Manager: }} This component is responsible for managing the data base, it is an interface between 
            the data base and the rest of the subsystems. The color assigned to this component is \textcolor[HTML]{8CC86E}{green}.
    \item \textcolor[HTML]{F0C832}{\textbf{Notification Manager: }} This component is responsible for managing the notifications; 
    This means sending notification, and serve the data to the front end to visualize the pending notification of a user. The color assigned to this component is \textcolor[HTML]{F0C832}{yellow}.
    \item \textcolor[HTML]{FF0000}{\textbf{Account Manager: }} This component manage accounts, in particular it handles sign in, log in, account changes and account deletion.
     The color assigned to this component is \textcolor[HTML]{FF0000}{red}.
    \item \textcolor[HTML]{64C8BE}{\textbf{LLM Prompter: }} This component is a simple library that provide an interface to prompt different LLMs. It's objective is to make the 
    implementation of the rest of the system agnostic from the specific LLM used. The color assigned to this component is \textcolor[HTML]{64C8BE}{aqua green}.
    \item \textcolor[HTML]{A0C8F0}{\textbf{Task Tree Navigator: }} This component is a simple library with some algorithm to navigate a tree data structure.
    In our specific case the tree represent the tasks and subtasks of an organization. Since this algorithms are used many
    times throughout the system they have been put in a specific component to make them reusable. The color assigned to this component is \textcolor[HTML]{A0C8F0}{light blue}.
    \item \textcolor[HTML]{FA9646}{\textbf{Report Manager: }} This component manage the reports, in particular it is responsible for generating report automatically (using an LLM) or manually (reminding users with a notification that they have to deliver a report).
    The component also allow users to set report schedule, so that the reports generation can be triggered automatically with a customizable frequency.
    The color assigned to this component is \textcolor[HTML]{FA9646}{orange}.
    \item \textcolor[HTML]{FF00FF}{\textbf{Organization Manager: }} This component allow an user to create an organization, or perform some action on his organizations, 
    like adding/removing a user, or paying the subscription to use the software. The color assigned to this component is \textcolor[HTML]{FF00FF}{purple}.
    \item \textcolor[HTML]{2682D5}{\textbf{Task Manager: }} This component allow users to interact with the tasks.
    In particular it is responsible to visualize, create, delete and edit tasks inside one organization. The color assigned to this component is \textcolor[HTML]{2682D5}{blue}.
    \item \textbf{Front End: } This component is responsible for the visualization of the data, and the interaction with the user. This component dose not have a specific color assigned to it, since it dose not provide any interface to other components.
    \item \textcolor[HTML]{E68CB4}{\textbf{External APIs: }} All interfaces provided by external APIs in the system (like authentication, payment, etc.) are painted with the color \textcolor[HTML]{E68CB4}{pink}.
\end{itemize}

\end{document}